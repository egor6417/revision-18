
\documentclass[12pt, letterpaper]{article}
\usepackage[utf8]{inputenc}
\usepackage[russian]{babel}
\usepackage{amsmath}
\usepackage{xfrac}

\title{Сочинения для ОГЭ 2018}
\author{Скорбенко Егор}
\date{Март 2018}

\begin{document}
\maketitle
\tableofcontents

\section {Что такое хороший человек?}
Что значит "хороший человек"? Хороший человек- это человек, который не является равнодушным и стремится помочь окружающим. Докажу свои рассуждения на примерах из прочитанного текста и из моего жизненного опыта.

В данном тексте хорошим человеком является военный. Он не прошел равнодушно мимо, когда увидел мальчика рядом с собакой. Военный подошел к маленькому человеку и поговорил с ним. Военнослужащий обьяснил ребенку, что быть собакой не имеет смысла, ведь самое важное в жизни- стать хорошим человеком. Об этом сказано в приложение №43: "Прежде всего надо быть хорошим человеком.".

В повести Г. Троепольского "Белый Бим черное ухо" главный герой, Иван Иванович, тоже является хорошим человеком. Он взял на воспитание щенка, который родился особенным, и из него вырос умный, воспитанный пёс.

Таким образом, быть хорошим человеком- значит всегда оставаться внимательным к нуждам других людей.

\section {Что такое совесть?}

Что такое совесть? Совесть- это чувство, которое испытывает человек, совершивший ошибку, из-за которой пострадали люди. Докажу свои рассуждения на примерах из прочитанного текста и из моего жизненного опыта.

В данном тексте угрызения совести испытывает герой-рассказчик, который был часовым. Когда военная часть, в которой служил этот солдат, стояла недалеко от передовой, комбат приказал выставить сторожевое охранение. Но люди очень устали, поэтому часовой пожалел людей и не выставил охрану. Когда с проверкой в часть приехал подполковник, он сделал выговор комбату за отсутствие охраны. Всю вину комбат взял на себя, поэтому герою-рассказчику стало очень стыдно за невыполнение приказа. Об этом сказано в предложении 24: "Виноват был я..."

В произведении А.С. Пушкина "Капитанская дочка" главный герой Петр Гринев поехал служить в Белогорскую крепость. По дороге он останавливался в трактире, где и познакомился с офицером Зуриным. Более опытный человек Зурин предложил научить играть в биллиард, при этом он выиграл у Гринева сто рублей. На утро Петру было очень стыдно перед Савельичем за то, что проиграл большие деньги. 

Таким образом, совесть необходима. Если бы у людей не было бы совести, наш мир бы очень быстро превратился в хаос. 
\section {Что такое настоящее искусство?}

Что такое настоящее искусство? Настоящее искусство- это искусство, которое вызывает различные эмоции у зрителей. Докажу свое рассуждение на примере из текста и из своего жизненного опыта.

В данном нам тексте настоящим искусством являет потрет Катри, который висел в хате Якова. Его впервые увидела Динка, и портрет ее поразил тем, что женщина на нем была как будто живая. Об этом сказано в предложении №10.

Недавно я посещал выставку картин художника Андрея Гросицкого, которая проходила в арт-пространстве "Винзавод". Все его работы отличаются яркими красками и необычной композицией. Особенно мне понравилась картина "Ликующие краски", на которой изображены краски, как бы "сбегающие" из серых тюбиков. Эта картина вызвала у меня чувство сожаления из-за того, что такого прекрасного художника не смогли по достоинству оценить в СССР, а ведь его картины замечательные.

Таким образом, настоящее искусство создается талантливыми людьми, и именно поэтому оно отличается от серой массы других произведений. 
\section{Что такое дружба?}
Что такое дружба? Дружба -  это близкие отношения, основанные прежде всего на понимании и поддержке. Настоящий друг всегда поможет и поддержит, не оставит тебя в трудной жизненной ситуации. Из-за этого ни один человек не может обойтись без друзей. Докажу свои рассуждения на примерах из прочитанного текста и из моего жизненного опыта.

В тексте Н. Татаринцева, в котором главным героем является Игорь Елисеев, Игорь оказался настоящим другом.  Когда весь класс пытался сбежать с урока, одноклассник Игоря, которого звали Петруха Васильев, остался. Ребятам это не понравилось, и они обвинили Петруху в трусости и предательстве. Однако главный герой текста догадался, из-за чего Петруха не сбежал: у матери Васильева был инфаркт, и она могла не перенести разбирательств, связанных с побегом. Игорь заступился за Петруху перед всем классом, тем самым поддержав Васильева. Елисеев пошёл против всех, ведь дружба с Петрухой для него важнее. 

% Приведу пример из рассказа "Уроки французского" % 	
Приведу пример из жизни. Мы дружим с другом уже почти десять лет, с первого класса. Когда мне плохо, я всегда могу быть уверен, что он меня выслушает и постарается помочь мне решить проблемы, которые так меня напрягают.

Эти примеры лишний раз подтверждают мое утверждение,  что важнее всего в дружбе-  поддержка и понимание.

\section{Что такое притворство?}

Что такое притворство? Притворство- это принятие образа, не соответсвующего реальности, чтобы обмануть других. Докажу свои рассуждения на примерах из прочитанного текста и из моего жизненного опыта.

В тексте Марка Твена, в котором главным героем является Том Сойер, Тому очень не хотелось идти в школе, и он притворился, будто у него болит зуб. После какого-то времени Том почувствовал, что его зуб и вправду болит. Такое самовнушение- отличительная особенность притворства.

Приведу пример из моего жизненного опыта.

\section{Что такое доброта?}
Что такое доброта? Доброта- зто стремление сделать нечто хорошее, помочь кому-то. Докажу свои рассуждения на примерах из прочитанного текста и из моего жизненного опыта.

В тексте Дмитрия Наркисовича Мамина-Сибиряка, в котором главным героем является охотник Емеля. Он проявил доброту по отношению к олененку, не убив его. Об этом сказано в предложении номер тридцать-шесть: "Точно что оборвалось в груди у старого Емели, и он опустил ружьё." Автор одобряет поступок Емели.

Примеру пример из моего жизненного опыта. В поэме великого русского писателя А. С. Пушкина «Евгений Онегин» протагонист Онегин убил своего знакомого на дуэли, и только с этого момента стал искренне несчастным, ибо он везде его приследовала совесть, ибо он совершил злой, недобрый поступок. Евгений мог дать путь своему чувству доброты и решить конфликт мирным путем, но он этого не сделал.
\end{document}