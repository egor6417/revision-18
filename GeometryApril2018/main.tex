\documentclass[12pt, letterpaper]{article}
\usepackage[utf8]{inputenc}
\usepackage[russian]{babel}
\usepackage{amsmath}
\usepackage{xfrac}

\title{Устный Зачёт по Геометрии}
\author{Скорбенко Егор}
\date{Апрель 2018}

\begin{document}
\maketitle
\tableofcontents

\section {Дайте определение угла между векторами, скалярного произведения векторов. Сформулируйте условие перпендикулярности. Докажите теорему о вычислении скалярного произведения векторов через их координаты. Выведите формулу для вычисления угла между векторами.}
\section {Сформулируйте и докажите свойства скалярного произведения векторов.}
\section {Дайте определение правильного многоугольника. Докажите, что около любого правильного многоугольника можно описать окружность.}
\section {Дайте определение правшгьного многоугольника. докажите. по в любой правильный многоугольник можно вписать окружность.}
\section {Выведите формулы для вычисления элементов правильного многоугольника (длина стороны, радиус вписанной окружности,  площадь) через радиус описанной окружности.}
\section {Выведите формулы для вычисления элементов правильного многоугольника (радиус описанной окружности, радиус вписанной окружности, площадь) через длину стороны.}
\section {Выведите формулы для вычисления радиуса описанной окружности и радиуса вписанной окружности в произвольном треугольнике}
\section {Дайте определения градуса и радиана. Выразите приближенное значение одного радиаиа в градусах. Выведите формулы для нахождения длины дуги через ее градусную меру и радианную.}
\section {Выведите формулы для нахождения площадей частей крута. }
\section {Сформулируйте свойства и признаки равнобедренной трапеции. Сформулируйте и допишите свойство равнобедренной трапеции с перпендикулярными диагоналями.}
\section {Дайте определение движения. сформулируйте общие свойства. Перечислите виды движений и их свойства.}
\section {Докажите теорему о произведении отрезков пересекающихся хорд окружности. докажите теорему о произведении отрезков секущей и квадрате касательной, проведенных из одной точки.}
\section {Сформулируйте и докажите теорему о величине угла между касательной и хордой.}
\section {Сформулируйте и докажите теоремы о величине углов между пересекающимися хордами, между секущими.}
\section {Сформулируйте и докажите теорему о сумме квадратов диагоналей параллелограмма.}
\section {Сформулируйте и докажите свойство диагоналей параллелограмма и формулу для вычисления длины медианы.}
\section {Сформулируйте признаки подобия треугольников. Докажите один из них по выбору.}
\section {Сформулируйте и докажите обобщенную теорему синусов.}
\section {Выведите формулы для нахождения пропорциональных отрезков в прямоугольном треугольнике. Выведите формулу для нахождения высоты прямоугольного треугольника через его стороны.}
\section {Выведите формулы для нахождения площади треугольника. (Не менее 4)}
\section {Выведите формулу площади произвольного четырехугольника и формулу площади дельтоида.}
\section {Сформулируйте и докажите теорему о центре окружности, вписанной в треугольник. Сформулируйте и докажите теорему о центре окружности, описанной около треугольника.}
\section {Сформулируйте и докажите свойство биссектрисы треугольника.}
\section {Сформулируйте и докажите свойство биссектрис параллелограмма.}
\section {Сформулируйте и докажите три свойства равнобедренной трапеции}
\section {Сформулируйте и докажите признаки прямоугольного треугольника. (Теорема, обратная теореме Пифагора и соотношение медианы и стороны, к которой она приведена.}
\section {Выведите формулы для нахождения радиусов вписанной и описанной окружностей через стороны правильных треугольника, квадрата и шестиугольника.}
\section {Дайте определение вписанного угла. Сформулируйте и докажите теорему о величине вписанного угла.}
\section {Сформулируйте и докажите свойство медиан в произвольном треугольнике.}
\section {Сформулируйте теорему Чевы. Сформулируйте и докажите теорему Менелая.}
\section {Сформулируйте и докажите свойства площадей треугольников с равными высотами, треугольников с равным углом и треугольников справными основаниями.}
\section {Сформулируйте и докажите свойства вписанного и описанного четырехугольника.}
\section {Найдите радиус вписанной и описанной окружностей прямоугольного треугольника.}
\section {Сформулируйте и докажите условия перпендикулярности и коллинеарность векторов через их координаты.} 

\end{document}