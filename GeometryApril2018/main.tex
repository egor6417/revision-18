\documentclass[12pt, letterpaper]{article}
\usepackage[utf8]{inputenc}
\usepackage[russian]{babel}
\usepackage{amsmath}
\usepackage{xfrac}
\usepackage[document]{ragged2e}

\title{Устный Зачёт по Геометрии}
\author{Скорбенко Егор}
\date{Апрель 2018}

\begin{document}
\maketitle
\tableofcontents

\section {Дайте определение угла между векторами, скалярного произведения векторов. Сформулируйте условие перпендикулярности. Докажите теорему о вычислении скалярного произведения векторов через их координаты. Выведите формулу для вычисления угла между векторами.}
\subsection{Определения}
Угол между векторами- угол между направлениями этих векторов.\\
Скалярным произведением двух векторов называется произведение их длин на косинус угла между ними. \\
Условие перпендикулярности- скалярное произведенеие ненулевых векторов равно нулю тогда и только тогда, когда эти векторы перпендикулярны. \\
\includegraphics[scale=0.3]{photo.jpg} \\
\subsection {Теорема о вычислении скалярного произведения векторов через их координаты.}
В прямоугольной системе координат скалярное произведение векторов $\vec{a}$ и $\vec{b}$ выражается формулой \\
\begin{center}
$\vec{a}*\vec{b}={x_1}{x_2}+{y_1}{y_2}$
\end{center}
\includegraphics[scale=0.3]{photo2.jpg} \\
\subsection {Вывод формулы для вычисления угла между векторами.}
Углом между двумя векторами, отложенными от одной точки, называется кратчайший угол, на который нужно повернуть один из векторов вокруг своего начала до положения сонаправленности с другим вектором. \\
Косинус угла между векторами равен скалярному произведению векторов, деленному на произведение модулей векторов. \\
\begin{center}
$ \cos a =\dfrac{\overline{a}*\overline{b}}{\dfrac{}{\left|a\right|}*{\left|b\right|}} $ \\
\end{center}


\section {Сформулируйте и докажите свойства скалярного произведения векторов.}
\includegraphics[scale=0.3]{photo.jpg} \\
\includegraphics[scale=0.3]{photo1.jpg} \\
\includegraphics[scale=0.3]{photo3.jpg} \\

\section {Дайте определение правильного многоугольника. Докажите, что около любого правильного многоугольника можно описать окружность.}
\subsection{Определения}
\textbf{Правильным многоугольником} называется выпуклый многоугльник, у которого все углы равны и все стороны равны. \\
Окружность называется описанной, если все вершины многоугольника лежат на данной окружности. \\
Около любого правильного многоугольника можно описать окружность, и притом только одну. \\
\subsection{Доказательство}
\includegraphics[scale=0.3]{photo4.jpg}

\section {Дайте определение правильного многоугольника. Докажите, что в любой правильный многоугольник можно вписать окружность.}
\subsection{Определения}
\textbf{Правильным многоугольником} называется выпуклый многоугльник, у которого все углы равны и все стороны равны. \\
Окружность называется вписанной, если все стороны многоугольника касаются данной окружности. \\
В любой правильный многоугольник можно вписать окружность, и притом только одну. \\
\subsection{Доказательство}
\includegraphics[scale=0.3]{photo5.jpg}

\section {Выведите формулы для вычисления элементов правильного многоугольника (длина стороны, радиус вписанной окружности,  площадь) через радиус описанной окружности.}
Докажем, что $S=\dfrac{1}{2}Pr$. \\
$S=\dfrac{1}{2}Pr=n*\dfrac{1}{2}*a*r=\dfrac{1}{2}*(n*a)r$ \\ 
\textbf{ЧТД.} \\
Далее:
$ a_n=2R*\sin \dfrac{180^{\circ}}{n} $ \\
$ r=R*\cos \dfrac{180^{\circ}}{n}$ \\
$ \angle A_1 = \dfrac{\alpha_n}{2}=\dfrac{n-2}{2n}*180^{\circ}=90^{\circ}-\dfrac{180^{\circ}}{n}. $ \\
$ a_n=2*A_1*H_1 $ \\
$ r=O*H_1 $ \\

\section {Выведите формулы для вычисления элементов правильного многоугольника (радиус описанной окружности, радиус вписанной окружности, площадь) через длину стороны.}
????????????

\section {Выведите формулы для вычисления радиуса описанной окружности и радиуса вписанной окружности в произвольном треугольнике}
???????

\section {Дайте определения градуса и радиана. Выразите приближенное значение одного радиаиа в градусах. Выведите формулы для нахождения длины дуги через ее градусную меру и радианную.}
\subsection{Определения}
Градус- единица измерения дуг и углов, равная 1/360 окружности. \\
Радиан- угол, соответствующий дуге, длина которой равна её радиусу. \\
1 Градус $\approx$ 0.0175 Радиан \\
\subsection{Формула через радианную меру}
$ l=\dfrac{\pi R}{180} n $ \\
\2subsection{Формула через градусную меру}
Градус -> радиан и аналогично пункту #2. \\ 


\section {Выведите формулы для нахождения площадей частей круга. }

\section {Сформулируйте свойства и признаки равнобедренной трапеции. Сформулируйте и допишите свойство равнобедренной трапеции с перпендикулярными диагоналями.}

\section {Дайте определение движения. сформулируйте общие свойства. Перечислите виды движений и их свойства.}
\subsection{Определения}
Движение плоскости- отображерние плоскости на себя, сохраняющее расстояния. \\
Центральная симметрия плоскости также является движением. \\
\subsection{Общие свойства}
1. Движение переводит прямую в прямую, а параллельные прямые в параллельные прямые. \\
2. Движение переводит полуплоскость с границей $\alpha $  в полуплоскость с границей $\phi$ , где $\phi$- образ прямой $\alpha$ \\
3. Движение сохраняет простое отношение трех точек прямой. \\
4. Движение переводит отрезок в отрезок. \\
5. Движение переводит луч в луч. \\
6. Движение переводит угол в равный ему угол. \\
7. Движение переводит взаимно перпендикулярные прямые во взаимно перпендикулярные прямые. \\

\subsection{Виды движений и их свойства}
1. \textbf{Симметрия осевая} \\
2. \textbf{Симметрия центральная} \\
3. \textbf{Симметрия зеркальная} \\
4. \textbf{Симметрия скользащая} \\
5. \textbf{Параллельный перенос} \\
6. \textbf{Поворот} \\




\section {Докажите теорему о произведении отрезков пересекающихся хорд окружности. докажите теорему о произведении отрезков секущей и квадрате касательной, проведенных из одной точки.}
\includegraphics[scale=0.3]{photo9.jpg} \\
\includegraphics[scale=0.3]{asset-3.jpg} \\

\section {Сформулируйте и докажите теорему о величине угла между касательной и хордой.}
\includegraphics[scale=0.3]{asset-2.png} \\

\section {Сформулируйте и докажите теоремы о величине углов между пересекающимися хордами, между секущими.}
\includegraphics[scale=0.3]{asset-3.jpg} \\

\section {Сформулируйте и докажите теорему о сумме квадратов диагоналей параллелограмма.}
???????

\section {Сформулируйте и докажите свойство диагоналей параллелограмма и формулу для вычисления длины медианы.}
????

\section {Сформулируйте признаки подобия треугольников. Докажите один из них по выбору.}
\subsection{Определения}
Два треугольника называются \textbf{подобными}, если их углы соответственно равны и стороны одного треугольника пропорциональны сходственным сторонам другого треугольника. \\
1. Если два угла одного треугольника соответственно равны двум углам другого, то такие треугольники подобны. \\
2. 2 стороны пропорциональны + угол равен \\
3. 3 стороны пропорциональны \\

\begin{center}
$ \dfrac{AB}{A_1 B_1} = \dfrac{BC}{B_1 C_1} = K $ \\ 
\end{center}

\subsection{Доказательство}
\includegraphics[scale=0.3]{photo8.jpg} \\

\section {Сформулируйте и докажите обобщенную теорему синусов.}
\includegraphics[scale=0.3]{photo6.jpg} \\
\includegraphics[scale=0.3]{photo7.jpg} \\

\section {Выведите формулы для нахождения пропорциональных отрезков в прямоугольном треугольнике. Выведите формулу для нахождения высоты прямоугольного треугольника через его стороны.}
??


\section {Выведите формулы для нахождения площади треугольника. (Не менее 4)}
\subsection {Стандартная}
$S=\dfrac{1}{2}ah $ \\
\textbf{Доказательство:} достроим до параллелограмма ABCD,\\
$\Delta ABC = \Delta DCB = \dfrac{1}{2}ABCD. $ \\

\subsection {Формула Герона}
\includegraphics[scale=1]{asset.png} \\
$S=\sqrt{(p(p-a)(p-b)(p-c))}$ \\
$p=\dfrac{P}{2}=\dfrac{a+b+c}{2}$ \\
\textbf{Доказательство:} \\
\begin{flushleft}
$S=\dfrac{1}{2}absin\phi$ => $S^2=\dfrac{1}{4}a^2b^2sin^2\phi=\dfrac{1}{4}a^2b^2(1-cos^2\phi).$ \\
$c^2=a^2+b^2-2abcos\phi$ => $cos\phi=\dfrac{a^2b^2-c^2}{2ab}=cos^2\phi$\\
$S^2=\dfrac{1}{4}a^2b^2(1-cos^2\phi)= $ \\
$ \dfrac{1}{4}a^2b^2(1-(\dfrac{a^2+b^2-c^2}{2ab})^2)=$ \\
$ \dfrac{1}{16}(4a^2b^2-(a^2+b^2-c^2)^2)= $ \\
$ \dfrac{1}{16}((a+b)^2-c^2)(c^2-(a-b)^2)= $ \\
$ \dfrac{1}{16}(a+b+c)(a+b-c)(c+a-b)(c-a+b)= $ \\
$ \dfrac{1}{16}(a+b+c)(a+b+c-2c)(c+a+b-2b)(a+b+c-2a)= $ \\ 
$ \dfrac{1}{16}2p(2p-2c)(2p-2b)(2p-2a)= p(p-c)(p-b)(p-a)$ \\
\end{flushleft}
\textbf{$ S=\sqrt{(p(p-a)(p-b)(p-c))} $ ЧТД.}

\subsection {Полупериметр и вписанная окружность}
$S=\dfrac{1}{2}Pr$ \\
\textbf{Доказательство:} \\
\begin{flushleft}
$S=\dfrac{1}{2}Pr=n*\dfrac{1}{2}*a*r=\dfrac{1}{2}*(n*a)r$ \\
\end{flushleft}

\subsection {Формула через синус}
$S=\dfrac{1}{2}ab*sin\alpha $ \\
\textbf{Доказательство:} \\
\begin{flushleft}
$h_a=b*\sin \alpha $ \\
$S=\dfrac{1}{2} a*h_a= \dfrac{1}{2}ab*\sin \alpha $ \\
\end{flushleft}


\section {Выведите формулу площади произвольного четырехугольника и формулу площади дельтоида.}
\section {Сформулируйте и докажите теорему о центре окружности, вписанной в треугольник. Сформулируйте и докажите теорему о центре окружности, описанной около треугольника.}
\section {Сформулируйте и докажите свойство биссектрисы треугольника.}
\section {Сформулируйте и докажите свойство биссектрис параллелограмма.}
\section {Сформулируйте и докажите три свойства равнобедренной трапеции}
\section {Сформулируйте и докажите признаки прямоугольного треугольника. (Теорема, обратная теореме Пифагора и соотношение медианы и стороны, к которой она приведена.}
ФШФШФШ
\section {Выведите формулы для нахождения радиусов вписанной и описанной окружностей через стороны правильных треугольника, квадрата и шестиугольника.}
ФОФ
\section {Дайте определение вписанного угла. Сформулируйте и докажите теорему о величине вписанного угла.}
ОО
\section {Сформулируйте и докажите свойство медиан в произвольном треугольнике.}
фофо

\section {Сформулируйте теорему Чевы. Сформулируйте и докажите теорему Менелая.}
\section {Сформулируйте и докажите свойства площадей треугольников с равными высотами, треугольников с равным углом и треугольников справными основаниями.}
\section {Сформулируйте и докажите свойства вписанного и описанного четырехугольника.}
\section {Найдите радиус вписанной и описанной окружностей прямоугольного треугольника.}
фофо
\section {Сформулируйте и докажите условия перпендикулярности и коллинеарность векторов через их координаты.} 
ффофо

\end{document}